%TITLE=Uhti-uhti uhkesti
%COMPOSER=eesti rahvaviis
%POET=A. Piirkivi
%LMARGIN=0.8in
%REFLMARGIN=1.1in
Uhti-uhti uhkesti,
viisk l\"aks Tartust Viljandi,
kaasas p\~ois ja \~olek\~ors,
\~olek\~ors kui s\"a\"ase\~ors.
:,: Ei saa \"ule Emaj\~oest,
h\"uva n\~ou on kallis t\~oest! :,:

Viisk siis \"utles targasti
oma seltsilistele:
{,,}Otsas h\"ada, otsas vaev.
P\~ois on meil kui loodud laev.
:,: K\~ors on mastiks k\~olbulik;
Mina olen laevnik.{``} :,:

P\~ois aga vastas p\~orinal,
v\"aga k\~ova k\~ori tal:
{,,}Viisu n\~ou on imelik:
laev see olgu laberik.
:,: Vaat'ke, kui ma veeren vees,
kohe oletegi sees!{``} :,:

K\~ors siis heitis pikali,
sillakene valmiski;
viisk aga h\"u\"udis: {,,}Sild on h\"a\"a!
Mina sammun \"ule pea!{``}
:,: Astus sammu, astus kaks:
Sillakene katki, praks! :,:

P\~onnadi, p\~onnadi, h\"uppas p\~ois,
naeris, mis ta naerda v\~ois!
{,,}Vaata, kuidas rumalad
uppusivad m\~olemad!{``}
:,: T\~ombas ennast hinge t\"ais
Ja, karplauhti, l\~ohki k\"ais! :,: