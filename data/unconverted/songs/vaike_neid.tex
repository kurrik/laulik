%TITLE=V\"aike neid, oota vaid
%COMPOSER=E. Taube
%POET=J. T\"atte
V\"aike neid, oota vaid,
veel on maid, kuhu said
antud hoiule mu armut\~ootused,
kuid kui merelt tulen taas,
h\"u\"uan r\~o\~omsalt hollarii
ja sul tasku pistan kingiks dollari.

Kui ma laeval olen t\"o\"ol,
sulle m\~otlen p\"aeval, \"o\"ol
ja su juustest leitud linti kannan v\"o\"ol,
kuid kui sadam neelab mind
ja jalge all on kindel pind,
ootab mind, ootab mind v\~o\~oras linn.

Linna tunda pole vaja,
seal, kus meremeeste maja,
seal on laulu-naeru kosta taevani,
siin on mehed, kel on kulda,
ja kui r\"o\"ovima ei tulda,
pidu p\"usida v\~oib varavalgeni.

Liiter viskit peseb maha
huultelt meresoola vaha,
piip on suus ja lahkub sealt vaid suudlusteks.
Piigad siin ei ole pahad,
kui on taskus suured rahad,
\"Uhe dollari vaid j\"atan truuduseks.

Igas mere\"a\"arses linnas,
kus on laevatatav pinnas,
T\"uki s\"udamest ma j\"atnud \~oitsema.
Kuid kui aastaring saab \"umber,
p\"o\"oran laevaotsa \"umber
ning ma koju s\~oidan, koju s\~oidan taas.

V\"aike neid, oota vaid\ldots