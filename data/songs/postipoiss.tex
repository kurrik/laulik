%TITLE=Postipoiss
%POET=J. Pori
%COMPOSER=A. Marcs
Palju aastaid l\"ainud m\"o\"oda sellest a'ast,
kui veel olemas ei olnud meie maal
ronge, autosid, jalgrattaid,
ega tehtud pikki matkaid
nagu t\"anap\"aeval k\~oikjal n\"{a}ha saad.

%REFRAIN
Uhke postipoiss s\~oitis kord maanteel,
k\"ulast k\"ulla, linnast linna viis ta tee.
K\~oikjal uudiseid t\~oi postipoiss kaugelt,
n\"u\"ud vaid m\"o\"odun'd ajast m\"alestus on see,
on see, on see\ldots

Talvel lumi tuiskas, tormas maantee peal,
juba kaugelt kostis aisakella h\"a\"al.
Soojas kasukas seal s\~oitis
m\~oni rikas linna kaupmees.
S\~oidust m\~onusasti habe h\"armas tal.

%REFRAIN
Uhke postipoiss s\~oitis kord maanteel \ldots

Kirju kallimalt ka postipoiss veel t\~oi,
vahel suudluse ka selle eest siis sai.
Mitu h\~oberaha taskus
selle vaeva \"ara tasus,
mida teekond pikk tall' kaasa tuua v\~ois.

%REFRAIN
Uhke postipoiss s\~oitis kord maanteel \ldots

Postijaam ja k\"ulak\~orts need \"uheskoos,
soojas kambris ikka tuju oli hoos.
Siis kui jalgu puhkas hobu,
peeti k\"ulmarohust lugu,
varsti s\~oit l\"aks lahti j\"alle t\"aies hoos.