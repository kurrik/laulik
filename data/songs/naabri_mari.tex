%TITLE=Naabri Mari
%POET=K. A. Hermann
%COMPOSER=eesti rahvaviis
%LMARGIN=0.9in
%REFLMARGIN=1.2in
Kuula, kui naabri Mari laulab,
n\~onda et \"o\"obik teda kuulab.
Ega l\~oo l\~o\~oritse---
naabri Mari laulab.
Ega \"o\"obik h\"a\"alitse---
naabri Mari laulab.
Laulab, laulab, laulab!

Mehed k\~oik j\"a\"avad m\~onutsema,
naised ei hakka s\~onutsema;
k\~oik nad tah'vad kuulata---
naabri Mari laulab.
Himu \"uhes kuuluta---
naabri Mari laulab.
Laulab, laulab, laulab!

L\~obusast naabri Marist neiut
loodavad m\~orsjaks saada peiud,
et saaks kogu linnuk'se,
kes nii kaunilt laulab;
korjab hingest pinnuk'se,
sest ta ikka laulab.
Laulab, laulab, laulab!