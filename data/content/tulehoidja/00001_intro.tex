\null
\vspace{0.5in}
\hrule height 0.05ex
\vspace{.1in}
\centerline{\footnotesize v\"aljaantud: J\"arvemetsal 2022.a.}
\centerline{\footnotesize v\"aljaandja: Eesti Gaid- ja Skautmalevad USAs.    }
\vspace{.1in}
\hrule height 0.05ex
\vspace{.25in}
Nimi:\dotfill
\vfill
\noindent{\footnotesize Koostanud: \bf{Mart Kuhn, Arne Roomann-Kurrik, Liina Sarapik}}\\
{\footnotesize Palju t\"anu: \bf{Juhan Aavik, EELK, Tiina Ets, Tiina Fischgrund, M. Hunt, Leena Kangro, Erni Kilm, hr. Kool, Juta K\~oiv, Inda K\~oiva, Katrin Laube, Leelo Linask, Madis Linask, mart@bronti.ms.ut.ee, Linold Milles, Meediagrupp S\"ud-Est ja ETV Suveprojekt, Kadri Munro, Kristiina Nielander, Tiiu N\~ommik, Andres Parmas, Aado Perandi, Andres Raudsepp, Reklaamiagentuur Hunt, Nutella Riisipuu, Katrin-Kaja Roomann, Aili Sarapik, Enn Sarapik, hr. Suu, hr. Tamkivi, Voldemar Tamman, Susanna Tomson, Gunnar Tamm, ja k\~oik, kes on kunagi laulikut koostanud vıi toimetada aidanud. }}\\
{\footnotesize Tarkvara: \bf{\LaTeX 5.3, lilypond 2.8.1, Python 2.4, Ubuntu Linux 6.06}}
\clearpage
\null
\vspace{2in}
\noindent {\it ,,Laul olgu l\"uhike v\~oi pikk, peaasi et on \~opetlik.``}
\cleardoublepage
{\samepage\raggedbottom
\raggedright
\sloppy
\centerline{ {\bf {\large Legend lakewoodi valgest liivast.}}}
\vspace{0.1in}
\hrule height 0.05ex
\vspace{0.1in}
\centerline{ {\em {\footnotesize nskm. Jyri Kork'i kirjutatud kogudest}}}}
\vspace{.05in}

Lakewoodi valge liiv on tagasihoidlik nagu s\~onajala \~ois.  Teda ei leidu laia juttu ajavas rahvahulgas Piprapoe leti ees.  Teda pole v\~oimalik n\"aha keskp\"aeva l\~o\~omavas p\"aikeses gaidide supluspaigas j\"arve kaldal.  Ja teda ei n\"ae iial need linnalapsed, kes kolmehobuj\~oulise taskulambiga \"o\"osel pimestavad k\~oiki vastutulejaid.  Lakewoodi valge liiv helendab \"uksikuil metsaradadel.  Ta muutub seda valgemaks, mida s\"ugavamaks s\"uveneb \"o\"o.  S\~obralikult juhib ta iga skaudipoissi ja gaidit\"udrukut, kes ainult kuu v\~oi t\"ahtede valgusel s\"aeb oma hiliseid samme.  Kuid k\~oige valgem on ta siiski talvistes laagrim\"alestustes.  Ja see Lakewoodi valge liiv on pealegi n\~oiutud liiv!  Sellest liivast v\~oib leida kulda.  Neid kullateri on loomulikult palju-palju raskem otsida, kui valget liiva ennast.  Aga ometi leitakse neid ja viiakse koju kaasa.

Mismoodi n\"aevad v\"alja need kullaterad?

\"Uks naerun\"aoline m\"alestus, millest emale ei r\"a\"agita, aga mis soojendab s\"udant kogu koolitalve.  \"Uks uus s\~oprus, s\~olmitud kahe teineteisest tuhandete miilide kaugusel elava eesti skaudipoisi vahel kes varjasid end samas kadakap\~o\~osas laagripolitsei ja gaidlaagri juhi eest.  \"Uks hiilga vahva laul, mida \~opiti vihmas suitseva ja susiseva l\~okke juures, kui juht l\"aks magama.

Jah, Lakewoodis on olemas valge liiv!

\hfill
Jyri Kork, nskm.